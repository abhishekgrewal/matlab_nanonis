% !TEX encoding = UTF-8 Unicode

% !TEX encoding = UTF-8 Unicode
\documentclass[a4paper, 12pt]{article}

\usepackage[T1]{fontenc}%encodage texte (Accents, ...) output
\usepackage[utf8]{inputenc}%encodage entrée


%		Bases Latex:
%
% Le texte s'écrit dans un environnement. L'environnement commence par un \begin{}
% et fini par \end{}. Il peut y avoir des environnements dans des environnements mais 
% ils ne doivent jamais se croiser:
% 
% \begin{A}\begin{B}\end{B}\end{A} => OK
% \begin{A}\begin{B}\end{A}\end{B} => NON!!!!!
% 
% L'environnement du document est "document", pour les maths: "displaymath", etc.
% 
% Le texte doit être structuré. La structure la plus commune est:
% \section{TITRE}, \subsection{TITRE}, \subsubsection{TITRE}
% une étoile évite la numérotation: \section*{}
% Le texte s'écrit simplement au dessous de ces definitions
% 
% Si les maths ne sont pas dans un environnement special, ils doivent être
% entourés de $ ou de $$. Par exemple pour écrire un mu, il faut écrire $\mu$
% 
% \def ou \newcommand permettent de définir des raccourcis. Préférez \newcommand 
% qui fait plus de vérifications. voila la syntaxe:
% \newcommand{\myCommand}[NbrArguments][Default value first argument]{What my command do}
% Dans la commande, on se réfère au premier(etc.) argument avec #1
% Les [] sont optionnels. Si on met un premier argument optionnel,
% on appelle la fonction avec \myCommand[first]{second}. Sinon, \myCommand{1}{2}
% 


\usepackage{graphicx}%Affichage d'images
\graphicspath{{../figures/}}%search for images in figures/
\usepackage[colorlinks,bookmarks=false,linkcolor=blue,urlcolor=blue]{hyperref}%links
\usepackage[all]{hypcap}%correct bug figures+link
\usepackage{gensymb}%Generic symbols (\degree, ...)
\usepackage{ifthen}% conditional statement
\usepackage{makeidx}%index
\usepackage[]{algorithm2e}%affichage algorithmes
\usepackage{caption}%Do caption job
\usepackage{subcaption}%If we want to use subfigures
\usepackage{textcomp} % se vuoi usare il \texttrademark


%ALGORITHM
%%%%%%%%%%%%%%%%%%%%%%%%%%%%%%%%%%%%%%%%%%%%%%%%%%%%%%%%%%%%%%
\RestyleAlgo{boxed} % I want box around algorithms
\newcommand{ \ba }{\begin{algorithm}[H]} %algortihm displays algorithms
\newcommand{ \ea }{\end{algorithm}}


%MATH
%%%%%%%%%%%%%%%%%%%%%%%%%%%%%%%%%%%%%%%%%%%%%%%%%%%%%%%%%%%%%%
% equations
\newcommand{ \be }{\begin{equation}} %maths numérotés
\newcommand{ \ee }{\end{equation}}
\newcommand{ \bdm }{\begin{displaymath}}%maths non numérotés
\newcommand{ \edm }{\end{displaymath}}


%tableaux
%%%%%%%%%%%%%%%%%%%%%%%%%%%%%%%%%%%%%%%%%%%%%%%%%%%%%%%%%%%%%%
\newcommand{\bt}[1]{\begin{table}[!h]\begin{center}\begin{tabular}{#1}\hline}
\newcommand{\et}[1]{\hline\end{tabular}\caption{#1}\end{center}\end{table}}


%figures
%%%%%%%%%%%%%%%%%%%%%%%%%%%%%%%%%%%%%%%%%%%%%%%%%%%%%%%%%%%%%%

\newenvironment{centeredFigure}[1][hptb]
{
\begin{figure}[#1]
\begin{center}
}{
\end{center}
\end{figure}
}

\newcommand{\multiFig}[3][hptb]{
\begin{centeredFigure}[#1]
#2
\caption{#3}
\end{centeredFigure}
}

\newcommand{\insertFig}[4][hptb]{%Par défaut, le premier argument est hp
	\begin{centeredFigure}[#1]
		\ifthenelse{\equal{#4}{}}{%Si le #4 est vide, 
			\includegraphics[width=\textwidth]{#2}
		}{
			\includegraphics[width=#4]{#2}
		}
		\caption{ \label{fig_#2} #3}
	\end{centeredFigure}
}


\newcommand{\subFig}[3][.45\textwidth]{
\begin{subfigure}{#1}
  \centering
  \includegraphics[width=\linewidth]{#2}
  \caption{\label{fig_#2}#3}
\end{subfigure}
}


\newcommand{\fig}[1]{Figure \ref{fig_#1}}% ref fait référence à un \label{}

%listes
%%%%%%%%%%%%%%%%%%%%%%%%%%%%%%%%%%%%%%%%%%%%%%%%%%%%%%%%%%%%%%
%\newcommand{\+}[1][{}]{
%\ifthenelse{\equal{#1}{}}{%Si le #1 est vide, on envoie \item
%\item
%}{
%\item[#1]
%}}
%Itemize
\newcommand{ \bi }{\begin{itemize}} %Itemize fait une liste avec des bullet, enumerate avec chiffres
\newcommand{ \ei }{\end{itemize}}
\newcommand{ \bis }{\bi \setlength{\itemsep}{-3pt}}%Small
%definitions
\newcommand{ \bdf }{\begin{description}}% liste de description
\newcommand{ \edf }{\end{description}}


%Liens
%%%%%%%%%%%%%%%%%%%%%%%%%%%%%%%%%%%%%%%%%%%%%%%%%%%%%%%%%%%%%%
\newcommand{\mail}[1]{{\href{mailto:#1}{#1}}} %Lien mail
\newcommand{\ftplink}[1]{{\href{ftp://#1}{#1}}}
\newcommand{\link}[1]{{\href{http://#1}{#1}}} %Lien Internet

%comment
%%%%%%%%%%%%%%%%%%%%%%%%%%%%%%%%%%%%%%%%%%%%%%%%%%%%%%%%%%%%%%
\newcommand{\cmt}[2]{
\if0#2%On compare 0 et #2 si ils sont égaux on affiche #1 
#1
\fi
}

%Detection of environnment
%%%%%%%%%%%%%%%%%%%%%%%%%%%%%%%%%%%%%%%%%%%%%%%%%%%%%%%%%%%%%%
\makeatletter %pour pouvoir utiliser le @ dans les commandes
\def\subs#1{ %on défini une commande nommée \subs avec 1 argument
   \def\@tempa{itemize} % on défini \@tempa comme étant itemize 
   %(Le \@ retiens un nom d'environnement)
   \ifx\@tempa\@currenvir %\@currenvir est l'environnement courrant
   % si c'est bon on fait un truc
    \ei
	\subsection{#1}
	\bi
	%fin du truc
    \else
    %Sinon on fait autre chose
      \subsection{#1}
      
   \fi
}
\makeatother % le @ redeviens normal

%index
%%%%%%%%%%%%%%%%%%%%%%%%%%%%%%%%%%%%%%%%%%%%%%%%%%%%%%%%%%%%%%
%\makeindex
\newcommand{\idx}[1]{#1\index{#1}}

\usepackage{amsmath}
\renewcommand{\familydefault}{\sfdefault}
\setlength\parindent{0pt}


\newcommand{\+}[1]{\item \textbf{#1}}
\newcommand{\nanonis}{\href{http://www.specs-zurich.com/en/home.html;jsessionid=FCD8A587EE447665C3F4A8CC374671EE}{Nanonis SPM Control System\texttrademark}}
\newcommand{\matlab}{MATLAB\texttrademark}

%Title
\title{NanoLib user guide}
%Utiliser \breakline plutôt que \\ dans du texte normal
%\author{Quentin Peter\\{\small \mail{qpeter@stud.phys.ethz.ch}}}
\author{
ETH Zurich\\
Laboratory of Solid State Physics\\\vspace{.5em}
Microstructure Research\\ 
{\small Danilo A. Zanin - \mail{dzanin@phys.ethz.ch}}\\
{\small Lorenzo G. De Pietro - \mail{depietro@phys.ethz.ch}}\\
{\small Quentin Peter - \mail{qpeter@stud.phys.ethz.ch}}
}

\date{\today}

\begin{document}
\maketitle
NanoLib library allows to open and analyze data generated by the \nanonis{} in \matlab.
The first version was developed in 2015 by Quentin Peter during its master thesis \emph{Spin Polarized Field Emission STM and Image Processing} in the Solid State Laboratory for \href{http://www.microstructure.ethz.ch}{Microstruture Research} at the ETH Zurich under the supervision of Dr. U. Ramsperger and L.G. De Pietro.
Some of the features of this library are still oriented to solve problems related to Quentin's thesis, e.g., \emph{scan\_type} field in the header structure (see \ref{sec:sxmFile}). 
In next versions these feature may be changed and generalized.\\

NanoLib library is divided in the package folders: \emph{+sxm}, \emph{+dat} and \emph{+utility}.

A function in a folder called \emph{+folder} can be called as \emph{folder.function}.\\

\newpage
\tableofcontents

%-------------------------------------------------------------------------------%
%  SXM FILES                                                                    %
%-------------------------------------------------------------------------------%
%!TEX root = matlabDoc.tex
\section{The +sxm package folder}

Images generated via the scanning interface of the \nanonis{} have extension \emph{file.sxm}. 
Once loaded, variable, from now on \emph{sxmFile}, is a structure divided in: \emph{i}) \textbf{header}, a structure containing all information present in the header of the \emph{file.sxm}, and \emph{ii}) \textbf{channels}, an array of channel structures containing data and information about every channel.
Both, \textbf{header} and \textbf{channels} can be called by
\begin{center}
\emph{sxmFile.header} \quad \text{and} \quad \emph{sxmfile.channels\{\#\}}
\end{center}
\# being the number of the channel. When only one channel is loaded one refers to the channel simply by \emph{sxmfile.channels}. 
More information about the substructure of header and channels is presented below.


\subsection{sxmFile: header and channels structures}
\label{sec:sxmFile}

The functions works with a structure that holds every relevant informations. Header and channels structure have following fields:

\bdf
\item[header] is a structure composed of:
  \bdf
  \item[scan\_file] name of the file
  \item[rec\_date] date of the scan
  \item[rec\_time] time of the scan
  \item[scan\_pixels] [nx;ny], number of pixels
  \item[scan\_range] [rx;ry], range [m]
  \item[scan\_offset] [ox;oy], offset [m]
  \item[scan\_angle] tilt angle of the scan
  \item[scan\_dir] 'up' or 'down'
  \item[bias] bias voltage [V]
  \item[scan\_type] 'STM', 'SEMPA', 'NFESEM', etc.
  \item[$\cdots$] Others informations extracted from the file
  \edf

\item[channels] is an array of channel structures composed of:
  \bdf
  \item[Direction] 'forward' or 'backward'
  \item[Unit] 'Z' or whatever the unit is
  \item[Name] The name of the channel
  \item[data] A $n\times m$ matrix of processed data
  \item[lineMedian] A $n\times 1$ matrix of raw line median
  \item[lineMean] A $n\times 1$ matrix of raw line mean
  \item[linePlane] A $n\times 1$ matrix of raw line mean linear fit
  \item[lineResidualSlope] a $1\times m$ matrix of processed column mean linear fit
  \item[lineStd] A $n\times 1$ matrix of processed line standard deviation
  \edf
\edf

To access the scan data on a structure named \emph{sxmFile}, one should type, e.g., \emph{sxmFile.header.rec\_date}.

%-------------------------------------------------------------------------------%
%  +load                                                                        %
%-------------------------------------------------------------------------------%
\subsection{+load}
This folder contains everything needed to load and process \emph{.sxm} files.
% ------------------------------ %
\subsubsection{loadsxm}
\bdf

\+{header = loadsxm(fn)} loads a file named \emph{fn.sxm} and returns the Header. This function is called by \emph{load.loadProcessedSxM} and \textbf{should not be called directly}.

\+{[header, data] = loadsxm(fn, i)} reads the channel \emph{i} and returns its \emph{data}.

\edf

\emph{.sxm} files are composed of an ascii header and of single precision binary data. They are separated by 0x1A 0x04 (SUB EOT).
This file is provided by \nanonis{} and loads a specified channel from a \emph{.sxm} file.
% ------------------------------ %
\subsubsection{processChannel}

\bdf
\+{channel = processChannel(channel, header)} Process the \emph{channel} as described below using the informations form \emph{header}. This function is called by \emph{load.loadProcessedXXX} and should not be called directly.

\+{channel = processChannel(channel,header,corrType)} If \emph{corrType} is set to 'Median', the median is used instead of the mean for lines corrections. If it is set to 'PlaneLineCorrection' a linear fit is used.  
\edf

The processing orientate and rotate the data so that all the images are comparable.
Everything that is removed is saved in the output structure to avoid loosing informations.

The mean value of the measurement under the conditions of each pixel must be extracted from the data. As there is drift and other instabilities, the mean value of the data is generally not a good value. The mean of each line is used instead, as the measurement conditions doesn't change too much during one line. Others possibility include the median or the mean plane. The mean plane along the line is also removed.

For STM, This offset is subtracted. For NFESEM and SEMPA, it is divided, as justified in the thesis.
% ------------------------------ %
\subsubsection{loadProcessedSxM}
\bdf
\+{file=loadProcessedSxM(fn)} loads and process all the channels of \emph{.sxm} file named \emph{fn}. The structure \emph{file} contains all the informations and is used in a large number of other functions.

\+{file=loadProcessedSxM(fn, chn)} only loads the channels whose numbers are in the array \emph{chn}

\+{file=loadProcessedSxM(fn, corrType)} If \emph{corrType} is set to 'MedianCorrection', the median is used instead of the mean for lines corrections. If it set to 'PlaneLineCorrection' a linear fit is used. 

\edf

The loading is done with \emph{load.loadsxm} and processing with \emph{load.processChannel}.

% \subsubsection{loadProcessedPar}
% \bdf
% \+{file=loadProcessedPar(fn)} loads and process the \emph{.par} file named \emph{fn}. The structure \emph{file} contains all the informations and is used in a large number of other functions.
%
% \+{file=loadProcessedPar(fn, corrType)} If \emph{corrType} is set to 'MedianCorrection', the median is used instead of the mean for lines corrections. If it set to 'PlaneLineCorrection' a linear fit is used.
%
% \edf
%
% The par data are composed of a \emph{.par} file that holds the header and of several \emph{.tfi} files that holds int 16 binary data for each channel.
%
% A header structure that match the \emph{.sxm} header structure is extracted from the \emph{.par} file, as well as infos about the Channels.

%-------------------------------------------------------------------------------%
%  +plot                                                                        %
%-------------------------------------------------------------------------------%
\subsection{+plot}
This package contains everything needed to plot the data.

\subsubsection{folder2png} 
\bdf
\+{folder2png(folderName)} finds every \emph{.par} and \emph{.sxm} files in \emph{folderName}, plot all relevant channels and saves the images in a \emph{image} folder.
\edf
% ------------------------------ %
\subsubsection{plotData}
\bdf
\+{[h, range] = plotData(data, name, unit, header)} plots the \emph{data} using informations from the \emph{header}. The figure title is deduced from \emph{name} and \emph{unit}. It returns the plot handle \emph{h} and the chosen range \emph{range}.
 
\+{[h, range] = plotData(data, name, unit, header, xoffset, yoffset)} adds an offset to the plot.
\edf

The range is 2 STD. If the data is STM, only the lines with low std are considered for the range.
% ------------------------------ %
\subsubsection{plotChannel}
\bdf
\+{[h, range] = plotChannel(channel, header)} plots the \emph{channel} using informations from the \emph{header}. It returns the plot handle \emph{h} and the chosen range \emph{range}.

\+{[h, range] = plotChannel(channel, header, xoffset, yoffset)} adds an offset to the plot.
\edf

It calls \emph{plot.plotData} on the channel data.
% ------------------------------ %
\subsubsection{plotFile}

\bdf
\+{[h, range] = plotFile(file, n)} plots the $n^{th}$ channel of \emph{file}. It returns the plot handle \emph{h} and the chosen range \emph{range}.

\+{[h, range] = plotFile(file, n, xoffset, yoffset)} adds an offset to the plot.
\edf

It calls \emph{plot.plotChannel}. 
% ------------------------------ %
\subsubsection{plotHistogram}
\bdf
\+{plotHistogram(data, range)} plots an histogram of \emph{data} and draw lines on the limit of \emph{range}. It removes the $.1\%$ most extreme values. 
\edf

%-------------------------------------------------------------------------------%
%  +op                                                                          %
%-------------------------------------------------------------------------------%
\subsection{+op}
This package contains various useful functions.
\subsubsection{combineChannel}
\bdf
\+{channel=combineChannel(file, name, chn, chw)} combined the channels \emph{chn} of the \emph{file} structure with weights \emph{chw} and return a new \emph{channel} with name \emph{name}.
\edf
% ------------------------------ %
\subsubsection{filterData}
\bdf
\+{[filtered, removed] = filterData(data, pixSize)} filters the \emph{data} with Fourier transform. The filtering keeps structures of approximatively \emph{pixSize} pixels. It returns the filtered data \emph{filtered} and the removed noise \emph{removed}.

\+{[filtered, removed] = filterData(data, pixSize, 'plotFFT', zoom)} additionally plots the Fourier plane. The optional variable \emph{zoom} has default value $8$ and is used to zoom in the Fourier plane.
\edf
% ------------------------------ %
\subsubsection{getOffset}
\bdf
\+{[offset, XC, centerOffset] = getOffset(img1, header1, img2, header2)} compares the images matrices \emph{img1} and \emph{img2} using informations from the two \emph{headeri} to find the most probable \emph{offset}. The units of \emph{offset} are from header.scan\_range. It correspond to the maximum of the cross correlation matrix \emph{XC}. The corresponding offset relative to the centre of the two images is returned in \emph{centerOffset}.

\+{[offset, XC, centerOffset] = getOffset(img1, header1, img2, header2, 'mask')} compares masks instead of images.
\edf
The offset is from the origin of the image, which is in a corner. The offset of the center is the centerOffset, but is less convenient to work with.
% ------------------------------ %
\subsubsection{getRadialFFT}
\bdf
\+{[wavelength, radial\_spectrum] =getRadialFFT(data)} Computes the \emph{radial spectrum} of the image saved in \emph{data} and the corresponding \emph{wavelength}. The wavelength unit is pixel.

\+{[wavelength, radial\_spectrum] =getRadialFFT(data,pixPerUnit)} Changes the wavelength unit with the number of pixels per units, \emph{pixPerUnit}.
\edf
This function is used to study the radial spectrum of an image computed from the FFT.
% ------------------------------ %
\subsubsection{getRadialNoise}
\bdf
\+{[noise\_fit, signal\_start, signal\_error, noise\_coeff] = getRadialNoise( wavelength, radial\_average)} tries to fit a noise from the data of \emph{getRadialFFT}. \emph{noise\_fit} is the detected noise. \emph{signal\_start} is the first position where the signal is detected. \emph{signal\_error} is the error caused by the discrete nature of the signal on \emph{signal\_start}. \emph{noise\_coeff} gives the power law coefficients for the first detected noise.

\+{[noise\_fit, signal\_start, signal\_error, noise\_coeff] = getRadialNoise( wavelength, radial\_average, maxNbrNoise)} Limits the number of noises to \emph{maxNbrNoise}. The default value is 10.
\edf
% ------------------------------ %
\subsubsection{getRange}
\bdf
\+{[xrange, yrange] = getRange(header)} extract the ranges \emph{xrange}, \emph{yrange} from \emph{header}.
\edf
% ------------------------------ %
\subsubsection{nanHighStd}
\bdf
\+{data = nanHighStd(data)} is useful for STM measurements. Usually the lines with very high std don't carry informations, and thus if a line has $std > 3 median$, it is set to nan.
\edf
% ------------------------------ %
\subsubsection{nanonisMap}
\bdf
\+{colorMap = nanonisMap(nPti)} is a color map function that generates a Nanonis color like mapping of \emph{nPti} number of colors. \emph{nPti} is an optional value. If not provided \emph{nPti} = 64 per default. 
\edf
% ------------------------------ %
\subsubsection{interpHighStd}
\bdf
\+{data = interpHighStd(data)} Removes the lines with high STD values and interpolates the missing values.
\edf
% ------------------------------ %
\subsubsection{interpPeaks}
\bdf
\+{data = interpPeaks(data)} Removes the data witch are too far from the mean and interpolates the missing values.
\edf

%-------------------------------------------------------------------------------%
%  +mask                                                                        %
%-------------------------------------------------------------------------------%
\subsection{+mask}
Theses functions are useful to compute threshold mask and apply them.
% ------------------------------ %
\subsubsection{applyMask}
\bdf
\+{applyMask(mask)} apply the boolean mask \emph{mask} to the current figure.

\+{applyMask(mask, color, alpha, xrange, yrange)} apply the boolean mask \emph{mask} in the range \emph{xrange}, \emph{yrange} with color \emph{color} and transparency \emph{alpha}. 
\edf

The ranges are vectors containing a start point and an end point. See MATLAB's \emph{image} documentation.
% ------------------------------ %
\subsubsection{getMask}
\bdf
\+{[maskUp, maskDown, flatData] = getMask(data, pixSize, prctUp, prctDown)}
 flatten and filter the \emph{data} before computing threshold masks. \emph{flatData} is the flattened and filtered data. \emph{maskUp} marks everithing above \emph{prctUp} and \emph{maskDown} below \emph{prctDown}. The filtering is done using \emph{op.filterData}, to which \emph{pixSize} is passed to keep features of this approximate size.

\+{[maskUp, maskDown, flatData] =  getMask(data, pixSize, prctUp, prctDown, 'plotFFT', zoom)}
Additionally passes \emph{'plotFFT',zoom} to \emph{op.filterData} to visualize the Fourier plane. \emph{zoom} is optional.
\edf

The flattening is done using sliding mean.

%-------------------------------------------------------------------------------%
%  +convolve2                                                                   %
%-------------------------------------------------------------------------------%
\subsection{+convolve2}
This in an improved version of MATLAB's conv2 matrix. It allows a better gestion of boundaries. 
It was downloaded from \href{http://www.mathworks.com/matlabcentral/fileexchange/22619-fast-2-d-convolution}{MATLAB file exchange}. See the license file.
%-------------------------------------------------------------------------------%
%  DAT FILES                                                                    %
%-------------------------------------------------------------------------------%
%!TEX root = matlabDoc.tex
\section{The +dat package folder}

Besides surface imaging \nanonis{} allows to store data measured by the physical channels. Data from the so called experiments are stored in a \emph{file.dat}.
Once loaded, variable, from now on \emph{datFile}, is a structure divided in: \emph{i}) \textbf{header}, a structure containing all information present in the header of the \emph{file.dat}, and \emph{ii}) \textbf{channels}, an array of channel structures containing data and information about every channel.
Both, \textbf{header} and \textbf{channels} can be called by
\begin{center}
\emph{datFile.header} \quad \text{and} \quad \emph{datfile.channels\{\#\}}
\end{center}
\# being the number of the channel. When only one channel is loaded one refers to the channel simply as \emph{datfile.channels}. 
More information about the substructure of header and channels is presented below.


\subsection{datFile: header and channels tructures}
\label{sec:sxmFile}

The functions works with a structure that holds every relevant informations. To access the scan data on a structure named \emph{expFile}, one should type \emph{expFile.header.rec\_date}. Header and channels structure have following fields:

\bdf
\item[header] is a structure composed of:
  \bdf
  \item[file] name of the file
  \item[path] path of the file
  \item[experiment] experiment name
  \item[rec\_date] date of the scan
  \item[rec\_time] time of the scan
  \item[points] number of experiment points
  \item[grid\_points] number of experiment repetition
  \item[channels] list of registered channels
  \item[list] is a $2 \times n$ list of string, $n$ being the number of lines in the text header. Lines in the \emph{header.list} are of the form \{'Key','data'\}, e.g., \{'rec\_date','22.08.2016'\}
  \item[$\cdots$] Others informations extracted from the file depending on the specific experiment
  \edf

\item[channels] is an array of channel structures composed of:
  \bdf
  \item[Direction] 'forward' or 'backward'
  \item[Unit] 'Z' or whatever the unit is
  \item[Name] The name of the channel
  \item[data] A $n\times m$ matrix of processed data, where $n$ is the number of points and $m$ is the loop number (default $m=1$)
  \edf
\edf
\textbf{The first channel, i.e. channel(1), is reserved to the sweep\_signal.}

%-------------------------------------------------------------------------------%
%  +load                                                                        %
%-------------------------------------------------------------------------------%
\subsection{+load}
This folder contains everything needed to load and process \emph{.dat} files.
% ------------------------------ %
\subsubsection{loaddat}
This file is provided by \nanonis{} and loads a specified channel from a \emph{file.dat}.
\bdf
\+{[header,data,channels]=loadDat(fn)} loads a file named \emph{fn.dat} and returns the \emph{header}, the \emph{data} and the \emph{channels} list.
 This function is called by \emph{load.loadProcessedDat} and \textbf{should not be called directly}.
\edf
% ------------------------------ %
\subsubsection{experiment\_$\ast$}
\emph{files.dat} are all characterized by a unique \textbf{experiment\_name}, that is saved in the first line of every \emph{.dat} file.
In the follow we refer to those \emph{files.dat} simply as \emph{experiments}.
Different \emph{experiments} have different headers and data characteristics.
Every \emph{experiment} have a specific function called \emph{experiment\_$\ast$}, \emph{$\ast$} being the name of the experiment.
\emph{experiment\_$\ast$} are called automatically by the \emph{loadProcessedDat} function as listed below.

\bdf
\+ {experiment\_name = experiment\_$\ast$('get experiment')} returns the \textbf{experiment\_name}.
\+ {header = experiment\_$\ast$('process header',header)} process the \emph{header} of the \emph{experiment}. 
The \emph{header} variable is result of the function .

\+ {[header,channels] = experiment\_$\ast$('process data',header,data)} stores data into the \emph{channels} structure described above. Where needed some additional processing are applied to the data. Header's information are adjusted accordingly.
\edf

Further \emph{experiments} can be implemented by simply defining a function called \emph{experiment\_newExperiment}. New \emph{experiment} functions \textbf{must} have the same structure described above and should be saved in the \emph{+load} package folder.

% ------------------------------ %
\subsubsection{getAllExperiments}
\bdf
\+{experiment\_list = getAllExperiments()} returns a $2 \times n$ list, where $n$ is the number of the function \emph{experiment\_$\ast$}.
In the first column is listed the unique name of the experiment saved in the \emph{+load} package folder.
In the second column compare the correspondent function, i.e., \emph{experiment\_$\ast$}.
\edf
This function is used by the function \emph{loadProcessedDat} when loading different \emph{experiments}.

% ------------------------------ %
\subsubsection{loadProcessedDat}
\emph{loadProcessedDat} loads a \emph{file.dat} calling the function \emph{loaddat}.
And process the \emph{header} and the \emph{data} according to the type of experiment by calling -- automatically -- the corresponding \emph{experiment\_$\ast$}.
\bdf
\+{file=loadProcessedDat()} ask for a \emph{fileName.dat} and load it.
\+{file=loadProcessedDat(fileName)} load the file named \emph{fileName.dat}.
\+{file=loadProcessedDat(fileName,pathName)} load the file named \emph{fileName.dat} at a given \emph{pathName}.
\edf

%-------------------------------------------------------------------------------%
%  +plotDat                                                                     %
%-------------------------------------------------------------------------------%
\subsection{+plotDat}
This package contains everything needed to plot the data.
% ------------------------------ %
\subsubsection{plotData}
\bdf
\+{hObject = plotData(data, name, unit, sweep\_channel,varargin)} plots the \emph{data} using according to the \emph{sweep\_channel}. The figure title is deduced from \emph{name} and \emph{unit}. It returns the plot handle \emph{hObject}. \emph{varargin} are the standard plot options. Additional options are provided.
\bi
\+ varargin = \{'xOffset', NUMBER \} shifts the x axis by the given offset
\+ varargin = \{'hideLabels'\} leaves all extra labels out.
\ei
\edf
% ------------------------------ %
\subsubsection{plotChannel}
\bdf
\+{hObject = plotChannel(channel,sweep\_channel,varargin)} plots the \emph{channel} using informations from the \emph{sweep\_channel}. It returns the plot handle  \emph{hObject}. It calls \emph{plot.plotData} on the channel data, \emph{varargin} are therefore the same as \emph{plot.plotData}.
\edf


% ------------------------------ %
\subsubsection{plotFile}

\bdf
\+{hObject = plotFile(file,channel\_numbers)} plots the $n^{th}$ channel of \emph{file}. \emph{channel\_numbers} may be a $n \times 1$ array. It returns the plot handle \emph{h}.

\+{hObject = plotFile(file,channel\_numbers,run\_numbers)} plots the $n^{th}$ repetition of the provided \emph{channel\_numbers} . Whenever an \emph{experiment} has more loops, repetitions are characterized by a \emph{run\_numbers}.
\edf

It calls \emph{plot.plotData} on the channel data.

%-------------------------------------------------------------------------------%
%  +opDat                                                                       %
%-------------------------------------------------------------------------------%
\subsection{+op}

To be done
% ------------------------------ %
% \subsubsection{getChannel}
% \bdf
% \+{channelNumber = getChannel(channels,channelNames)} returns all channel numbers where \emph{header.ChannelList} matches the \emph{channelNames}. \emph{channelNames} can be either a single string or a list of strings.
%
% \+{channelNumber = getChannel(channels,channelNames,direction)} returns only the channel number where \emph{channels.Direction} matches the \emph{direction}, too.
% \edf


%-------------------------------------------------------------------------------%
%  DAT FILES                                                                    %
%-------------------------------------------------------------------------------%
%!TEX root = matlabDoc.tex
\section{+utility}

The package folder \emph{+utility} contains generic functions, which can be used when analyzing both \emph{sxmFiles} and \emph{datFiles}
% ------------------------------ %
\subsubsection{combineChannel}
\bdf
\+{channel=combineChannel(file, name, chn, chw)} combined the channels \emph{chn} of the \emph{file} structure with weights \emph{chw} and return a new \emph{channel} with name \emph{name}.
\edf
% ------------------------------ %
\subsubsection{getAlphaColor}
\bdf
\+{outputRGB = getAlphaColor(inputRGB,alpha)} returns \emph{alpha\%} lighter \emph{inputRGB} color.

\+{outputRGB = getAlphaColor(inputRGB,alpha,'dark')} returns \emph{alpha\%} darker \emph{inputRGB} color.
\edf
% ------------------------------ %
\subsubsection{getChannel}
\bdf
\+{channelNumber = getChannel(channels,channelNames)} returns all channel numbers where \emph{channels.Name} matches the \emph{channelNames}. \emph{channelNames} can be either a single string or a list of strings.

\+{channelNumber = getChannel(channels,channelNames,direction)} returns only the channel number where \emph{channels.Direction} matches the \emph{direction}, too.
\edf
% ------------------------------ %
\subsubsection{getColor}
\bdf
\+{[color,colorScale] = getColor(x,xRange)} computes the ratio between a value \emph{x} and the \emph{xRange} (2 by 1 array). It returns the \emph{color} --  and the \emph{colorScale}  -- according to a predefined color map. \emph{Jet} is the default color map. 

\+{[color,colorScale] = getColor(x,xRange,mapping)} allows to provide e specified \emph{mapping} other than the default, i.e. \emph{jet}. \emph{mapping} should be a color map function, e.g. \emph{hsv}, \emph{parula}, \emph{hot}, \emph{summer}m \emph{autumn}. Note that, since \emph{mapping} is an argument of functions \emph{getColor}, it must be called by function handle ``at'' -- @.
\edf

%-------------------------------------------------------------------------------%
%  EXAMPLE                                                                      %
%-------------------------------------------------------------------------------%
%!TEX root = matlabDoc.tex
\section{Examples} 
\label{sec::examples}

NanoLib library comes with few example showing the basics usage of the library. Some files are also provided in the directory \emph{Files}.
Below a list of all examples with a short explanation.
% ------------------------------ %
\subsection{SxM\_Example}
\bdf
\item[example\_open\_SxM] shows different ways to load a \emph{file.sxm}.
\item[example\_process\_option] shows different ways to process a file while loading a \emph{file.sxm}.
\item[example\_get\_drift] detect XY-offset between two different \emph{sxmFiles}.
\item[example\_mask] generates a mask of a \emph{sxmFile} and apply on the original image.
\item[example\_RadialFFT] applies \emph{op.getRadialFFT} and plots some interesting quantities.
\edf
\subsection{dat\_Example}
\bdf
\item[example\_open\_Dat] shows different ways to load some \emph{experiment.dat}.
\edf
\end{document}

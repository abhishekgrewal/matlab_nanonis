% !TEX encoding = UTF-8 Unicode

% !TEX encoding = UTF-8 Unicode
\documentclass[a4paper, 12pt]{article}

\usepackage[T1]{fontenc}%encodage texte (Accents, ...) output
\usepackage[utf8]{inputenc}%encodage entrée


%		Bases Latex:
%
% Le texte s'écrit dans un environnement. L'environnement commence par un \begin{}
% et fini par \end{}. Il peut y avoir des environnements dans des environnements mais 
% ils ne doivent jamais se croiser:
% 
% \begin{A}\begin{B}\end{B}\end{A} => OK
% \begin{A}\begin{B}\end{A}\end{B} => NON!!!!!
% 
% L'environnement du document est "document", pour les maths: "displaymath", etc.
% 
% Le texte doit être structuré. La structure la plus commune est:
% \section{TITRE}, \subsection{TITRE}, \subsubsection{TITRE}
% une étoile évite la numérotation: \section*{}
% Le texte s'écrit simplement au dessous de ces definitions
% 
% Si les maths ne sont pas dans un environnement special, ils doivent être
% entourés de $ ou de $$. Par exemple pour écrire un mu, il faut écrire $\mu$
% 
% \def ou \newcommand permettent de définir des raccourcis. Préférez \newcommand 
% qui fait plus de vérifications. voila la syntaxe:
% \newcommand{\myCommand}[NbrArguments][Default value first argument]{What my command do}
% Dans la commande, on se réfère au premier(etc.) argument avec #1
% Les [] sont optionnels. Si on met un premier argument optionnel,
% on appelle la fonction avec \myCommand[first]{second}. Sinon, \myCommand{1}{2}
% 


\usepackage{graphicx}%Affichage d'images
\graphicspath{{../figures/}}%search for images in figures/
\usepackage[colorlinks,bookmarks=false,linkcolor=blue,urlcolor=blue]{hyperref}%links
\usepackage[all]{hypcap}%correct bug figures+link
\usepackage{gensymb}%Generic symbols (\degree, ...)
\usepackage{ifthen}% conditional statement
\usepackage{makeidx}%index
\usepackage[]{algorithm2e}%affichage algorithmes
\usepackage{caption}%Do caption job
\usepackage{subcaption}%If we want to use subfigures
\usepackage{textcomp} % se vuoi usare il \texttrademark


%ALGORITHM
%%%%%%%%%%%%%%%%%%%%%%%%%%%%%%%%%%%%%%%%%%%%%%%%%%%%%%%%%%%%%%
\RestyleAlgo{boxed} % I want box around algorithms
\newcommand{ \ba }{\begin{algorithm}[H]} %algortihm displays algorithms
\newcommand{ \ea }{\end{algorithm}}


%MATH
%%%%%%%%%%%%%%%%%%%%%%%%%%%%%%%%%%%%%%%%%%%%%%%%%%%%%%%%%%%%%%
% equations
\newcommand{ \be }{\begin{equation}} %maths numérotés
\newcommand{ \ee }{\end{equation}}
\newcommand{ \bdm }{\begin{displaymath}}%maths non numérotés
\newcommand{ \edm }{\end{displaymath}}


%tableaux
%%%%%%%%%%%%%%%%%%%%%%%%%%%%%%%%%%%%%%%%%%%%%%%%%%%%%%%%%%%%%%
\newcommand{\bt}[1]{\begin{table}[!h]\begin{center}\begin{tabular}{#1}\hline}
\newcommand{\et}[1]{\hline\end{tabular}\caption{#1}\end{center}\end{table}}


%figures
%%%%%%%%%%%%%%%%%%%%%%%%%%%%%%%%%%%%%%%%%%%%%%%%%%%%%%%%%%%%%%

\newenvironment{centeredFigure}[1][hptb]
{
\begin{figure}[#1]
\begin{center}
}{
\end{center}
\end{figure}
}

\newcommand{\multiFig}[3][hptb]{
\begin{centeredFigure}[#1]
#2
\caption{#3}
\end{centeredFigure}
}

\newcommand{\insertFig}[4][hptb]{%Par défaut, le premier argument est hp
	\begin{centeredFigure}[#1]
		\ifthenelse{\equal{#4}{}}{%Si le #4 est vide, 
			\includegraphics[width=\textwidth]{#2}
		}{
			\includegraphics[width=#4]{#2}
		}
		\caption{ \label{fig_#2} #3}
	\end{centeredFigure}
}


\newcommand{\subFig}[3][.45\textwidth]{
\begin{subfigure}{#1}
  \centering
  \includegraphics[width=\linewidth]{#2}
  \caption{\label{fig_#2}#3}
\end{subfigure}
}


\newcommand{\fig}[1]{Figure \ref{fig_#1}}% ref fait référence à un \label{}

%listes
%%%%%%%%%%%%%%%%%%%%%%%%%%%%%%%%%%%%%%%%%%%%%%%%%%%%%%%%%%%%%%
%\newcommand{\+}[1][{}]{
%\ifthenelse{\equal{#1}{}}{%Si le #1 est vide, on envoie \item
%\item
%}{
%\item[#1]
%}}
%Itemize
\newcommand{ \bi }{\begin{itemize}} %Itemize fait une liste avec des bullet, enumerate avec chiffres
\newcommand{ \ei }{\end{itemize}}
\newcommand{ \bis }{\bi \setlength{\itemsep}{-3pt}}%Small
%definitions
\newcommand{ \bdf }{\begin{description}}% liste de description
\newcommand{ \edf }{\end{description}}


%Liens
%%%%%%%%%%%%%%%%%%%%%%%%%%%%%%%%%%%%%%%%%%%%%%%%%%%%%%%%%%%%%%
\newcommand{\mail}[1]{{\href{mailto:#1}{#1}}} %Lien mail
\newcommand{\ftplink}[1]{{\href{ftp://#1}{#1}}}
\newcommand{\link}[1]{{\href{http://#1}{#1}}} %Lien Internet

%comment
%%%%%%%%%%%%%%%%%%%%%%%%%%%%%%%%%%%%%%%%%%%%%%%%%%%%%%%%%%%%%%
\newcommand{\cmt}[2]{
\if0#2%On compare 0 et #2 si ils sont égaux on affiche #1 
#1
\fi
}

%Detection of environnment
%%%%%%%%%%%%%%%%%%%%%%%%%%%%%%%%%%%%%%%%%%%%%%%%%%%%%%%%%%%%%%
\makeatletter %pour pouvoir utiliser le @ dans les commandes
\def\subs#1{ %on défini une commande nommée \subs avec 1 argument
   \def\@tempa{itemize} % on défini \@tempa comme étant itemize 
   %(Le \@ retiens un nom d'environnement)
   \ifx\@tempa\@currenvir %\@currenvir est l'environnement courrant
   % si c'est bon on fait un truc
    \ei
	\subsection{#1}
	\bi
	%fin du truc
    \else
    %Sinon on fait autre chose
      \subsection{#1}
      
   \fi
}
\makeatother % le @ redeviens normal

%index
%%%%%%%%%%%%%%%%%%%%%%%%%%%%%%%%%%%%%%%%%%%%%%%%%%%%%%%%%%%%%%
%\makeindex
\newcommand{\idx}[1]{#1\index{#1}}

\usepackage{amsmath}
\renewcommand{\familydefault}{\sfdefault}
\setlength\parindent{0pt}


\newcommand{\+}[1]{\item \textbf{#1}}

%Title
\title{Matlab Files Documentation}
%Utiliser \breakline plutôt que \\ dans du texte normal
\author{Quentin Peter\\{\small \mail{qpeter@stud.phys.ethz.ch}}}
\date{\today}

\begin{document}
\maketitle
\tableofcontents
\newpage

\section*{Package Folders}
A function in a folder called \emph{+folder} can be called as \emph{folder.function} . This folder is called a package folder. In the following discussion, the section names with a + refers to a package folder.

\section{STM\_SEM}
\subsection{Image Structure}

The functions works with a structure that holds every relevant informations. To access the scan date on a structure named \emph{stmFile}, one should type \emph{stmFile.header.rec\_date}. The structure has the following fields:

\bdf
\item[header] is a structure composed of:
\bdf

\item[scan\_file] The name of the file
\item[rec\_date] The date of the scan
\item[rec\_time] The time of the scan
\item[scan\_pixels] [nx;ny], the number of pixels
\item[scan\_range] [rx;ry], the range [m]
\item[scan\_offset] [ox;oy], the offset [m]
\item[scan\_angle] The tilt angle of the scan
\item[scan\_dir] 'up' or 'down'
\item[bias] The bias voltage [V]
\item[scan\_type] 'STM', 'SEMPA', 'NFESEM', etc.
\item[$\cdots$] Others informations extracted from the file
\edf

\item[channels] is an array of channel structures composed of:
\bdf
\item[Direction] 'forward' or 'backward'
\item[Unit] 'Z' or whatever the unit is
\item[Name] The name of the channel
\item[data] A $n\times m$ matrix of processed data
\item[lineMedian] A $n\times 1$ matrix of raw line median
\item[lineMean] A $n\times 1$ matrix of raw line mean
\item[linePlane] A $n\times 1$ matrix of raw line mean linear fit
\item[lineResidualSlope] a $1\times m$ matrix of processed column mean linear fit
\item[lineStd] A $n\times 1$ matrix of processed line standard deviation
\edf

\edf
\subsection{+convolve2}
This in an improved version of MATLAB's conv2 matrix. It allows a better gestion of boundaries. 
It was downloaded from \href{http://www.mathworks.com/matlabcentral/fileexchange/22619-fast-2-d-convolution}{MATLAB file exchange}. See the license file.

\subsection{+load}
This folder contains everything needed to load and process \emph{.sxm} and \emph{.par} files.

\subsubsection{loadsxm}
\bdf

\+{header = loadsxm(fn)} loads the \emph{.sxm} file \emph{fn} and returns the Header. This function is called by \emph{load.loadProcessedXXX} and should not be called directly.

\+{[header, data] = loadsxm(fn, i)} reads the channel \emph{i} and returns its \emph{data}.

\edf

\emph{.sxm} files are composed of an ascii header and of single precision binary data. They are separated by 0x1A 0x04 (SUB EOT).

This file is provided by NANONIS and loads a specified channel from a \emph{.sxm} file.

\subsubsection{processChannel}

\bdf
\+{channel = processChannel(channel, header)} Process the \emph{channel} as described below using the informations form \emph{header}. This function is called by \emph{load.loadProcessedXXX} and should not be called directly.

\+{channel = processChannel(channel,header,corrType)} If \emph{corrType} is set to 'Median', the median is used instead of the mean for lines corrections. If it is set to 'PlaneLineCorrection' a linear fit is used.  
\edf

The processing orientate and rotate the data so that all the images are comparable.
Everything that is removed is saved in the output structure to avoid loosing informations.

The mean value of the measurement under the conditions of each pixel must be extracted from the data. As there is drift and other instabilities, the mean value of the data is generally not a good value. The mean of each line is used instead, as the measurement conditions doesn't change too much during one line. Others possibility include the median or the mean plane. The mean plane along the line is also removed.

For STM, This offset is subtracted. For NFESEM and SEMPA, it is divided, as justified in the thesis.

\subsubsection{loadProcessedSxM}
\bdf
\+{file=loadProcessedSxM(fn)} loads and process all the channels of \emph{.sxm} file named \emph{fn}. The structure \emph{file} contains all the informations and is used in a large number of other functions.

\+{file=loadProcessedSxM(fn, chn)} only loads the channels whose numbers are in the array \emph{chn}

\+{file=loadProcessedSxM(fn, corrType)} If \emph{corrType} is set to 'MedianCorrection', the median is used instead of the mean for lines corrections. If it set to 'PlaneLineCorrection' a linear fit is used. 

\edf

The loading is done with \emph{load.loadsxm} and processing with \emph{load.processChannel}.

\subsubsection{loadProcessedPar}
\bdf
\+{file=loadProcessedPar(fn)} loads and process the \emph{.par} file named \emph{fn}. The structure \emph{file} contains all the informations and is used in a large number of other functions.

\+{file=loadProcessedPar(fn, corrType)} If \emph{corrType} is set to 'MedianCorrection', the median is used instead of the mean for lines corrections. If it set to 'PlaneLineCorrection' a linear fit is used. 

\edf

The par data are composed of a \emph{.par} file that holds the header and of several \emph{.tfi} files that holds int 16 binary data for each channel. 

A header structure that match the .sxm header structure is extracted from the \emph{.par} file, as well as infos about the Channels.
\subsection{+mask}
Theses functions are useful to compute threshold mask and apply them.
\subsubsection{applyMask}
\bdf
\+{applyMask(mask)} apply the boolean mask \emph{mask} to the current figure. 
\+{applyMask(mask, color, alpha, xrange, yrange)} apply the boolean mask \emph{mask} in the range \emph{xrange}, \emph{yrange} with color \emph{color} and transparency \emph{alpha}. 
\edf

The ranges are vectors containing a start point and an end point. See MATLAB's \emph{image} documentation.

\subsubsection{getMask}
\bdf
\+{[maskUp, maskDown, flatData] = getMask(data, pixSize, prctUp, prctDown)}
 flatten and filter the \emph{data} before computing threshold masks. \emph{flatData} is the flattened and filtered data. \emph{maskUp} marks everithing above \emph{prctUp} and \emph{maskDown} below \emph{prctDown}. The filtering is done using \emph{op.filterData}, to which \emph{pixSize} is passed to keep features of this approximate size.

\+{[maskUp, maskDown, flatData] =  getMask(data, pixSize, prctUp, prctDown, 'plotFFT', zoom)}
Additionally passes \emph{'plotFFT',zoom} to \emph{op.filterData} to visualize the Fourier plane. \emph{zoom} is optional.
\edf

The flattening is done using sliding mean.

\subsection{+op}
This package contains various useful functions.
\subsubsection{combineChannel}
\bdf
\+{channel=combineChannel(file, name, chn, chw)} combined the channels \emph{chn} of the \emph{file} structure with weights \emph{chw} and return a new \emph{channel} with name \emph{name}.
\edf

\subsubsection{filterData}
\bdf
\+{[filtered, removed] = filterData(data, pixSize)} filters the \emph{data} with Fourier transform. The filtering keeps structures of approximatively \emph{pixSize} pixels. It returns the filtered data \emph{filtered} and the removed noise \emph{removed}.

\+{[filtered, removed] = filterData(data, pixSize, 'plotFFT', zoom)} additionally plots the Fourier plane. The optional variable \emph{zoom} has default value $8$ and is used to zoom in the Fourier plane.

\edf

\subsubsection{getOffset}
\bdf
\+{[offset, XC, centerOffset] = getOffset(img1, header1, img2, header2)} compares the images matrices \emph{img1} and \emph{img2} using informations from the two \emph{headeri} to find the most probable \emph{offset}. The units of \emph{offset} are from header.scan\_range. It correspond to the maximum of the cross correlation matrix \emph{XC}. The corresponding offset relative to the centre of the two images is returned in \emph{centerOffset}.  


\+{[offset, XC, centerOffset] = getOffset(img1, header1, img2, header2, 'mask')} compares masks instead of images.
\edf

The offset is from the origin of the image, which is in a corner. The offset of the center is the centerOffset, but is less convenient to work with.

\subsubsection{getRadialFFT}

\bdf
\+{[wavelength, radial\_spectrum] =getRadialFFT(data)} Computes the \emph{radial spectrum} of the image saved in \emph{data} and the corresponding \emph{wavelength}. The wavelength unit is pixel.
\+{[wavelength, radial\_spectrum] =getRadialFFT(data,pixPerUnit)} Changes the wavelength unit with the number of pixels per units, \emph{pixPerUnit}.
\edf

This function is used to study the radial spectrum of an image computed from the FFT.

\subsubsection{getRadialNoise}

\bdf
\+{[noise\_fit, signal\_start, signal\_error, noise\_coeff] = getRadialNoise( wavelength, radial\_average)} tries to fit a noise from the data of \emph{getRadialFFT}. \emph{noise\_fit} is the detected noise. \emph{signal\_start} is the first position where the signal is detected. \emph{signal\_error} is the error caused by the discrete nature of the signal on \emph{signal\_start}. \emph{noise\_coeff} gives the power law coefficients for the first detected noise.

\+{[noise\_fit, signal\_start, signal\_error, noise\_coeff] = getRadialNoise( wavelength, radial\_average, maxNbrNoise)} Limits the number of noises to \emph{maxNbrNoise}. The default value is 10.

\edf


\subsubsection{getRange}
\bdf
\+{[xrange, yrange] = getRange(header)} extract the ranges \emph{xrange}, \emph{yrange} from \emph{header}.
\edf

\subsubsection{nanHighStd}
\bdf
\+{data = nanHighStd(data)} is useful for STM measurements. Usually the lines with very high std don't carry informations, and thus if a line has $std > 3 median$, it is set to nan.
\edf
\subsubsection{interpHighStd}
\bdf
\+{data = interpHighStd(data)} Removes the lines with high STD values and interpolates the missing values.
\edf
\subsubsection{interpPeaks}
\bdf
\+{data = interpPeaks(data)} Removes the data witch are too far from the mean and interpolates the missing values.
\edf

\subsection{+plot}
This package contains everything needed to plot the data.

\subsubsection{folder2png}
\bdf
\+{folder2png(folderName)} finds every \emph{.par} and \emph{.sxm} files in \emph{folderName}, plot all relevant channels and saves the images in a \emph{image} folder.
\edf

\subsubsection{plotData}
\bdf
\+{[h, range] = plotData(data, name, unit, header)} plots the \emph{data} using informations from the \emph{header}. The figure title is deduced from \emph{name} and \emph{unit}. It returns the plot handle \emph{h} and the chosen range \emph{range}.
 
\+{[h, range] = plotData(data, name, unit, header, xoffset, yoffset)} adds an offset to the plot.
\edf

The range is 2 STD. If the data is STM, only the lines with low std are considered for the range.

\subsubsection{plotChannel}
\bdf
\+{[h, range] = plotChannel(channel, header)} plots the \emph{channel} using informations from the \emph{header}. It returns the plot handle \emph{h} and the chosen range \emph{range}.

\+{[h, range] = plotChannel(channel, header, xoffset, yoffset)} adds an offset to the plot.
\edf

It calls \emph{plot.plotData} on the channel data.

\subsubsection{plotFile}


\bdf
\+{[h, range] = plotFile(file, n)} plots the $n^{th}$ channel of \emph{file}. It returns the plot handle \emph{h} and the chosen range \emph{range}.

\+{[h, range] = plotFile(file, n, xoffset, yoffset)} adds an offset to the plot.
\edf

It calls \emph{plot.plotChannel}. 

\subsubsection{plotHistogram}
\bdf
\+{plotHistogram(data, range)} plots an histogram of \emph{data} and draw lines on the limit of \emph{range}. It removes the $.1\%$ most extreme values. 
\edf

\subsection{Tests}

\subsubsection{testDrift}
This script tests the XY-offset detection.

\subsubsection{testMask}
This script tests the mask generation, application, and drift detection.

\subsubsection{testMask2}
This script tests the mask generation, application, and drift detection.

\subsubsection{testSEM}
Test the SEM processing and plotting

\subsubsection{testSTM}
Test the STM processing and plotting

\subsubsection{testRadialFFT}
Test \emph{op.getRadialFFT} and plots some interesting quantities.

\subsubsection{testPar}
Test PAR files loading.

\subsubsection{testMedian}
Test different ways to load a file.

\subsection{Scripts}

\subsubsection{createImages}
Script to call \emph{plot.folder2png} on every folder inside a folder.

\subsubsection{stdVsNe}
Script to study the effect of the number of electrons on the standard deviation. The normalised variance found in the hysteresis calculations is also displayed.

\subsubsection{SeriesFFTSpectrum}
Scripts that deduces the resolution of a series of image from the Fourier transform.

\subsubsection{MultiSeriesFFTSpectrum}
Scripts that deduces the resolution of a series of series of image from the Fourier transform.

\subsubsection{STMFFTSpectrum}
Script used to compare STM and SEM images.

\subsection{Old}
Some old files that are not useful.

\end{document}
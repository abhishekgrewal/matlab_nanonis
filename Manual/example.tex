%!TEX root = nanoLib.tex
\section{Examples} 
\label{sec::examples}

NanoLib library comes with few example showing the basics usage of the library. Some files are also provided in the directory \emph{Files}.
Below a list of all examples with a short explanation.
% ------------------------------ %
\subsection{SxM\_Example}
\label{sxm::example}

\subsubsection*{example\_open\_SxM} shows different ways to load a \emph{file.sxm}.
\subsubsection*{example\_process\_option} shows different ways to process a file while loading a \emph{file.sxm}.
\subsubsection*{example\_plot\_SxM} shows different ways to load a \emph{file.sxm}.
\subsubsection*{example\_get\_drift} detect XY-offset between two different \emph{sxmFiles}.
\subsubsection*{example\_mask} generates a mask of a \emph{sxmFile} and apply on the original image.
\subsubsection*{example\_RadialFFT} applies \emph{op.getRadialFFT} and plots some interesting quantities.

% ------------------------------ %
\subsection{SxM\_viewer}
\label{sxm::viewer}
The SXM viewer allows to open sxm files and have a fast overview of all channels stored within a SXM image.

\subsubsection*{Setup}
In order to use the SXM viewer you need to set local paths for the NanoLib and the path to directory where images are saved.
The first time you run the viewer you will be asked to find paths. 
Information will be saved in a file \emph{localSettings.txt} in the same directory  where \emph{SXM.m} is with the following informations:\\
\begin{minipage}{\textwidth}
	\setlength{\parindent}{15pt}
	\texttt{nanoLib	$\sim$/MATLAB/matlab\_nanonis/NanoLib}\\
	\indent \texttt{dataPath	$\sim$/Nanonis/Data}
\end{minipage}

\subsubsection*{Usage}
\begin{enumerate}
	\item Run SXM viewer main file: \\ \mcode{SXM}
	\item Select the process type in the popupmenu (default = Raw)
	\item Press open button and search for the directory where measurements are;
	\item Press items in the list box in order to let them appear in a new figure;
	\item Press on plotted channels to export them on a new figure.
\end{enumerate}

% ------------------------------ %
\subsection{Dat\_Example}
\bdf
\item[examples\_Dat] shows different ways to load some \emph{experiment.dat}.
\edf

% ------------------------------ %
\subsection{Dat\_viewer}
\label{dat::viewer}
The DAT viewer allows to open dat files and have a fast overview of all channels stored within a dat file.

\subsubsection*{Setup}
In order to use the DAT viewer you need to set local paths for the NanoLib and the path to directory where images are saved.
The first time you run the viewer you will be asked to find paths. 
Information will be saved in a file \emph{localSettings.txt} in the same directory  where \emph{DAT.m} is with the following informations:\\
\begin{minipage}{\textwidth}
	\setlength{\parindent}{15pt}
	\texttt{nanoLib	$\sim$/MATLAB/matlab\_nanonis/NanoLib}\\
	\indent \texttt{dataPath	$\sim$/Nanonis/Data}
\end{minipage}

\subsubsection*{Usage}
\begin{enumerate}
	\item Run SXM viewer main file: \\ \mcode{DAT}
	\item Press open button and search for the directory where measurements are;
	\item Press items in the list box in order to let them appear in a new figure;
	\item Press on plotted channels to export them on a new figure.
	\item Whenever the button ‘hold exported’ is active (blue capital letters), all exported channels will be inserted in the same figure.
\end{enumerate}

% ------------------------------ %
\subsection{NanoLib\_micro}
\label{nanolib::user}

Contains user defined examples in order to integrate user defined experiments which are defined as \emph{file.dat}.
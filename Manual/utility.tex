%!TEX root = matlabDoc.tex
\section{+utility}

The package folder \emph{+utility} contains generic functions, which can be used when analyzing both \emph{sxmFiles} and \emph{datFiles}
% ------------------------------ %
\subsubsection{combineChannel}
\bdf
\+{channel=combineChannel(file, name, chn, chw)} combined the channels \emph{chn} of the \emph{file} structure with weights \emph{chw} and return a new \emph{channel} with name \emph{name}.
\edf
% ------------------------------ %
\subsubsection{getAlphaColor}
\bdf
\+{outputRGB = getAlphaColor(inputRGB,alpha)} returns \emph{alpha\%} lighter \emph{inputRGB} color.

\+{outputRGB = getAlphaColor(inputRGB,alpha,'dark')} returns \emph{alpha\%} darker \emph{inputRGB} color.
\edf
% ------------------------------ %
\subsubsection{getChannel}
\bdf
\+{channelNumber = getChannel(channels,channelNames)} returns all channel numbers where \emph{channels.Name} matches the \emph{channelNames}. \emph{channelNames} can be either a single string or a list of strings.

\+{channelNumber = getChannel(channels,channelNames,direction)} returns only the channel number where \emph{channels.Direction} matches the \emph{direction}, too.
\edf
% ------------------------------ %
\subsubsection{getColor}
\bdf
\+{[color,colorScale] = getColor(x,xRange)} computes the ratio between a value \emph{x} and the \emph{xRange} (2 by 1 array). It returns the \emph{color} --  and the \emph{colorScale}  -- according to a predefined color map. \emph{Jet} is the default color map. 

\+{[color,colorScale] = getColor(x,xRange,mapping)} allows to provide e specified \emph{mapping} other than the default, i.e. \emph{jet}. \emph{mapping} should be a color map function, e.g. \emph{hsv}, \emph{parula}, \emph{hot}, \emph{summer}m \emph{autumn}. Note that, since \emph{mapping} is an argument of functions \emph{getColor}, it must be called by function handle ``at'' -- @.
\edf
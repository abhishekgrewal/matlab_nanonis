%!TEX root = matlabDoc.tex
\section{The +dat package folder}

Besides surface imaging \nanonis{} allows to store data measured by the physical channels. Data from the so called experiments are stored in a \emph{file.dat}.
Once loaded, variable, from now on \emph{datFile}, is a structure divided in: \emph{i}) \textbf{header}, a structure containing all information present in the header of the \emph{file.dat}, and \emph{ii}) \textbf{channels}, an array of channel structures containing data and information about every channel.
Both, \textbf{header} and \textbf{channels} can be called by
\begin{center}
\emph{datFile.header} \quad \text{and} \quad \emph{datfile.channels\{\#\}}
\end{center}
\# being the number of the channel. When only one channel is loaded one refers to the channel simply as \emph{datfile.channels}. 
More information about the substructure of header and channels is presented below.


\subsection{sxmFile: header and channels tructures}
\label{sec:sxmFile}

The functions works with a structure that holds every relevant informations. To access the scan data on a structure named \emph{expFile}, one should type \emph{expFile.header.rec\_date}. Header and channels structure have following fields:

\bdf
\item[header] is a structure composed of:
  \bdf
  \item[file] name of the file
  \item[path] path of the file
  \item[experiment] experiment name
  \item[sweep\_signal] signal that is varied during the experiment, it can be also the time
  \item[rec\_date] date of the scan
  \item[rec\_time] time of the scan
  \item[points] number of experiment points
  \item[grid\_points] number of experiment repetition
  \item[channels] list of registered channels
  \item[list] is a $2 \times n$ list of string, $n$ being the number of lines in the text header. Lines in the \emph{header.list} are of the form \{'Key','data'\}, e.g., \{'rec\_date','22.08.2016'\}
  \item[$\cdots$] Others informations extracted from the file depending on the specific experiment
  \edf

\item[channels] is an array of channel structures composed of:
  \bdf
  \item[Direction] 'forward' or 'backward'
  \item[Unit] 'Z' or whatever the unit is
  \item[Name] The name of the channel
  \item[data] A $n\times m$ matrix of processed data, where $n$ is the number of points and $m$ is the loop number (default $m=1$)
  \edf
\edf
\textbf{The first channel, i.e. channel(1), is reserved to the sweep\_signal.}

%-------------------------------------------------------------------------------%
%  +load                                                                        %
%-------------------------------------------------------------------------------%
\subsection{+load}
This folder contains everything needed to load and process \emph{.dat} files.
% ------------------------------ %
\subsubsection{experiment\_$\ast$}
\emph{files.dat} are all characterized by a unique \textbf{experiment\_name}, that is saved in the first line of every \emph{.dat} file.
In the follow we refer to those \emph{files.dat} simply as \emph{experiments}.
Different \emph{experiments} have different headers and data characteristics.
Every \emph{experiment} have a specific function called \emph{experiment\_$\ast$}, \emph{$\ast$} being the name of the experiment.
\emph{experiment\_$\ast$} are called automatically by the \emph{loadDat} function as listed below.

\bdf
\+ {experiment\_name = experiment\_$\ast$('get experiment')} returns the \textbf{experiment\_name}.
\+ {header = experiment\_$\ast$('get header',header,datasForKey)} returns the complete \emph{header} of the \emph{experiment}. 
The input variable \emph{header} contains only the variables \emph{experiment} and \emph{list}.
  \bdf
  \+ {data = datasForKey(key)} is function returns the \emph{data} according to a specific \emph{key} as stored in the variable \emph{header.list}.
  \edf  
\+ {[header,channels] = experiment\_$\ast$('process data',header,data)} stores data into the \emph{channels} structure described above. Where needed some additional processing are applied to the data. Header's information are adjusted accordingly.
\edf

Further \emph{experiments} can be implemented by simply defining a function called \emph{experiment\_newExperiment}. New \emph{experiment} functions \textbf{must} have the same structure described above and should be saved in the \emph{+load} package folder.

% ------------------------------ %
\subsubsection{getAllExperiments}
\bdf
\+{experiment\_list = getAllExperiments()} returns a $2 \times n$ list, where $n$ is the number of the function \emph{experiment\_$\ast$}.
In the first column is listed the unique name of the experiment saved in the \emph{+load} package folder.
In the second column compare the correspondent function, i.e., \emph{experiment\_$\ast$}.
\edf
This function is used by the function \emph{loadDat} when loading different \emph{experiments}.


% ------------------------------ %
\subsubsection{loadDat}
By mean of the \emph{experiment} structure described in the two previous sections, \emph{loadDat} automatically recognize the type of \emph{experiment} and load it.
\bdf
\+{file=loadDat()} ask for a \emph{fileName.dat} and load it.
\+{file=loadDat(fileName)} load the file named \emph{fileName.dat}.
\+{file=loadDat(fileName,pathName)} load the file named \emph{fileName.dat} at a given \emph{pathName}.
\edf

%-------------------------------------------------------------------------------%
%  +plotDat                                                                     %
%-------------------------------------------------------------------------------%
\subsection{+plotDat}
This package contains everything needed to plot the data.
% ------------------------------ %
\subsubsection{plotData}
\bdf
\+{hObject = plotData(data, name, unit, sweep\_channel,varargin)} plots the \emph{data} using according to the \emph{sweep\_channel}. The figure title is deduced from \emph{name} and \emph{unit}. It returns the plot handle \emph{hObject}. \emph{varargin} are the standard plot options. Additional options are provided.
\bi
\+ varargin = \{'xOffset', NUMBER \} shifts the x axis by the given offset
\+ varargin = \{'hideLabels'\} leaves all extra labels out.
\ei
\edf
% ------------------------------ %
\subsubsection{plotChannel}
\bdf
\+{hObject = plotChannel(channel,sweep\_channel,varargin)} plots the \emph{channel} using informations from the \emph{sweep\_channel}. It returns the plot handle  \emph{hObject}. It calls \emph{plot.plotData} on the channel data, \emph{varargin} are therefore the same as \emph{plot.plotData}.
\edf


% ------------------------------ %
\subsubsection{plotFile}

\bdf
\+{hObject = plotFile(file,channel\_numbers)} plots the $n^{th}$ channel of \emph{file}. \emph{channel\_numbers} may be a $n \times 1$ array. It returns the plot handle \emph{h}.

\+{hObject = plotFile(file,channel\_numbers,run\_numbers)} plots the $n^{th}$ repetition of the provided \emph{channel\_numbers} . Whenever an \emph{experiment} has more loops, repetitions are characterized by a \emph{run\_numbers}.
\edf

It calls \emph{plot.plotData} on the channel data.

%-------------------------------------------------------------------------------%
%  +opDat                                                                       %
%-------------------------------------------------------------------------------%
\subsection{+op}

To be done
% ------------------------------ %
% \subsubsection{getChannel}
% \bdf
% \+{channelNumber = getChannel(channels,channelNames)} returns all channel numbers where \emph{header.ChannelList} matches the \emph{channelNames}. \emph{channelNames} can be either a single string or a list of strings.
%
% \+{channelNumber = getChannel(channels,channelNames,direction)} returns only the channel number where \emph{channels.Direction} matches the \emph{direction}, too.
% \edf

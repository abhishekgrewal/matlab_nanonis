% !TEX encoding = UTF-8 Unicode
\documentclass[a4paper, 12pt]{article}

\usepackage[T1]{fontenc}%encodage texte (Accents, ...) output
\usepackage[utf8]{inputenc}%encodage entrée


%		Bases Latex:
%
% Le texte s'écrit dans un environnement. L'environnement commence par un \begin{}
% et fini par \end{}. Il peut y avoir des environnements dans des environnements mais 
% ils ne doivent jamais se croiser:
% 
% \begin{A}\begin{B}\end{B}\end{A} => OK
% \begin{A}\begin{B}\end{A}\end{B} => NON!!!!!
% 
% L'environnement du document est "document", pour les maths: "displaymath", etc.
% 
% Le texte doit être structuré. La structure la plus commune est:
% \section{TITRE}, \subsection{TITRE}, \subsubsection{TITRE}
% une étoile évite la numérotation: \section*{}
% Le texte s'écrit simplement au dessous de ces definitions
% 
% Si les maths ne sont pas dans un environnement special, ils doivent être
% entourés de $ ou de $$. Par exemple pour écrire un mu, il faut écrire $\mu$
% 
% \def ou \newcommand permettent de définir des raccourcis. Préférez \newcommand 
% qui fait plus de vérifications. voila la syntaxe:
% \newcommand{\myCommand}[NbrArguments][Default value first argument]{What my command do}
% Dans la commande, on se réfère au premier(etc.) argument avec #1
% Les [] sont optionnels. Si on met un premier argument optionnel,
% on appelle la fonction avec \myCommand[first]{second}. Sinon, \myCommand{1}{2}
% 


\usepackage{graphicx}%Affichage d'images
\graphicspath{{../figures/}}%search for images in figures/
\usepackage[colorlinks,bookmarks=false,linkcolor=blue,urlcolor=blue]{hyperref}%links
\usepackage[all]{hypcap}%correct bug figures+link
\usepackage{gensymb}%Generic symbols (\degree, ...)
\usepackage{ifthen}% conditional statement
\usepackage{makeidx}%index
\usepackage[]{algorithm2e}%affichage algorithmes
\usepackage{caption}%Do caption job
\usepackage{subcaption}%If we want to use subfigures
\usepackage{textcomp} % se vuoi usare il \texttrademark


%ALGORITHM
%%%%%%%%%%%%%%%%%%%%%%%%%%%%%%%%%%%%%%%%%%%%%%%%%%%%%%%%%%%%%%
\RestyleAlgo{boxed} % I want box around algorithms
\newcommand{ \ba }{\begin{algorithm}[H]} %algortihm displays algorithms
\newcommand{ \ea }{\end{algorithm}}


%MATH
%%%%%%%%%%%%%%%%%%%%%%%%%%%%%%%%%%%%%%%%%%%%%%%%%%%%%%%%%%%%%%
% equations
\newcommand{ \be }{\begin{equation}} %maths numérotés
\newcommand{ \ee }{\end{equation}}
\newcommand{ \bdm }{\begin{displaymath}}%maths non numérotés
\newcommand{ \edm }{\end{displaymath}}


%tableaux
%%%%%%%%%%%%%%%%%%%%%%%%%%%%%%%%%%%%%%%%%%%%%%%%%%%%%%%%%%%%%%
\newcommand{\bt}[1]{\begin{table}[!h]\begin{center}\begin{tabular}{#1}\hline}
\newcommand{\et}[1]{\hline\end{tabular}\caption{#1}\end{center}\end{table}}


%figures
%%%%%%%%%%%%%%%%%%%%%%%%%%%%%%%%%%%%%%%%%%%%%%%%%%%%%%%%%%%%%%

\newenvironment{centeredFigure}[1][hptb]
{
\begin{figure}[#1]
\begin{center}
}{
\end{center}
\end{figure}
}

\newcommand{\multiFig}[3][hptb]{
\begin{centeredFigure}[#1]
#2
\caption{#3}
\end{centeredFigure}
}

\newcommand{\insertFig}[4][hptb]{%Par défaut, le premier argument est hp
	\begin{centeredFigure}[#1]
		\ifthenelse{\equal{#4}{}}{%Si le #4 est vide, 
			\includegraphics[width=\textwidth]{#2}
		}{
			\includegraphics[width=#4]{#2}
		}
		\caption{ \label{fig_#2} #3}
	\end{centeredFigure}
}


\newcommand{\subFig}[3][.45\textwidth]{
\begin{subfigure}{#1}
  \centering
  \includegraphics[width=\linewidth]{#2}
  \caption{\label{fig_#2}#3}
\end{subfigure}
}


\newcommand{\fig}[1]{Figure \ref{fig_#1}}% ref fait référence à un \label{}

%listes
%%%%%%%%%%%%%%%%%%%%%%%%%%%%%%%%%%%%%%%%%%%%%%%%%%%%%%%%%%%%%%
%\newcommand{\+}[1][{}]{
%\ifthenelse{\equal{#1}{}}{%Si le #1 est vide, on envoie \item
%\item
%}{
%\item[#1]
%}}
%Itemize
\newcommand{ \bi }{\begin{itemize}} %Itemize fait une liste avec des bullet, enumerate avec chiffres
\newcommand{ \ei }{\end{itemize}}
\newcommand{ \bis }{\bi \setlength{\itemsep}{-3pt}}%Small
%definitions
\newcommand{ \bdf }{\begin{description}}% liste de description
\newcommand{ \edf }{\end{description}}


%Liens
%%%%%%%%%%%%%%%%%%%%%%%%%%%%%%%%%%%%%%%%%%%%%%%%%%%%%%%%%%%%%%
\newcommand{\mail}[1]{{\href{mailto:#1}{#1}}} %Lien mail
\newcommand{\ftplink}[1]{{\href{ftp://#1}{#1}}}
\newcommand{\link}[1]{{\href{http://#1}{#1}}} %Lien Internet

%comment
%%%%%%%%%%%%%%%%%%%%%%%%%%%%%%%%%%%%%%%%%%%%%%%%%%%%%%%%%%%%%%
\newcommand{\cmt}[2]{
\if0#2%On compare 0 et #2 si ils sont égaux on affiche #1 
#1
\fi
}

%Detection of environnment
%%%%%%%%%%%%%%%%%%%%%%%%%%%%%%%%%%%%%%%%%%%%%%%%%%%%%%%%%%%%%%
\makeatletter %pour pouvoir utiliser le @ dans les commandes
\def\subs#1{ %on défini une commande nommée \subs avec 1 argument
   \def\@tempa{itemize} % on défini \@tempa comme étant itemize 
   %(Le \@ retiens un nom d'environnement)
   \ifx\@tempa\@currenvir %\@currenvir est l'environnement courrant
   % si c'est bon on fait un truc
    \ei
	\subsection{#1}
	\bi
	%fin du truc
    \else
    %Sinon on fait autre chose
      \subsection{#1}
      
   \fi
}
\makeatother % le @ redeviens normal

%index
%%%%%%%%%%%%%%%%%%%%%%%%%%%%%%%%%%%%%%%%%%%%%%%%%%%%%%%%%%%%%%
%\makeindex
\newcommand{\idx}[1]{#1\index{#1}}
